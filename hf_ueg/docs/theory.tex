\documentclass[12pt]{article}

\usepackage{amsmath}
\usepackage{physics}
\usepackage{graphicx}
\usepackage{hyperref}
\usepackage{listings} % Required for insertion of code
\usepackage{color} % Required for custom colors


% Define custom colors
\definecolor{codegreen}{rgb}{0,0.6,0}
\definecolor{codegray}{rgb}{0.5,0.5,0.5}
\definecolor{codepurple}{rgb}{0.58,0,0.82}
\definecolor{backcolour}{rgb}{0.95,0.95,0.92}

% Setup the style for code listings
\lstdefinestyle{mystyle}{
    backgroundcolor=\color{backcolour},   
    commentstyle=\color{codegreen},
    keywordstyle=\color{magenta},
    numberstyle=\tiny\color{codegray},
    stringstyle=\color{codepurple},
    basicstyle=\ttfamily\footnotesize,
    breakatwhitespace=false,         
    breaklines=true,                 
    captionpos=b,                    
    keepspaces=true,                 
    numbers=left,                    
    numbersep=5pt,                  
    showspaces=false,                
    showstringspaces=false,
    showtabs=false,                  
    tabsize=2
}

% Activate the style
\lstset{style=mystyle}

\author{Patryk Kozlowski}
\title{Hartree-Fock for the uniform electron gas}
\date{\today}
\begin{document}
\maketitle
\section{Bases}
We are working with a basis of plane waves that must be normalized. We do this by enforcing the following condition:
\begin{equation}
    \int_{\Omega } d\mathbf{r} e^{i\mathbf{k}\cdot\mathbf{r}} e^{-i\mathbf{k}\cdot\mathbf{r}} = \int_{\Omega }d\mathbf{r} = L^3 \rightarrow \phi_\mu(\mathbf{r}) = \frac{1}{\sqrt{\Omega }} e^{i\mathbf{k}_\mu\cdot\mathbf{r}}
\end{equation}
We also know that different plane waves with different wave vectors, but that satisfy the same boundary condition, must be orthogonal, so:
\begin{equation}
    \int_{\Omega } d\mathbf{r} \phi^*_\mu(\mathbf{r}) \phi_\nu(\mathbf{r}) = \delta_{\mu\nu}
\end{equation}

\section{Hamiltonian}
% In second quantization, the Hamiltonian can be given by the sum of one and two-electron integrals:
% \begin{equation}
%     \hat{H} = \sum_{\mu\nu} h_{\mu\nu} a^\dagger_\mu a_\nu + \frac{1}{2} \sum_{\mu\lambda\nu\sigma} g_{\mu\lambda\nu\sigma} a^\dagger_\mu a^\dagger_\nu a_\sigma a_\lambda
% \end{equation}
% where $h_{\mu\nu}$ is the one-electron integral and $g_{\mu\lambda\nu\sigma}$ is the two-electron integral. We start by simplifying the operator strings using Wick's theorem. 
% \begin{equation}
%    a^\dagger_\mu a_\nu \rightarrow \langle a^\dagger_\mu a_\nu \rangle = \delta_{\mu\nu}
% \end{equation}


% This was a trivial application of Wick's theorem leading to the observation that the only contribution to the one-electron term occurs for diagonal basis functions
% \begin{equation}
%     \sum_{\mu\nu} h_{\mu\nu} a^\dagger_\mu a_\nu = \sum_\mu h_{\mu\mu}= \sum_{\mu} \int d\mathbf{r} \phi^*_\mu(\mathbf{r}) \left( -\frac{1}{2} \nabla^2 \right) \phi_\mu(\mathbf{r}) = \sum_{\mu} \frac{\mathbf{k}_\mu^2}{2}
% \end{equation}
% Next, we do the same for the two-electron term
% \begin{equation}
%     a^\dagger_\mu a^\dagger_\nu a_\sigma a_\lambda \rightarrow \langle a^\dagger_\mu a_\lambda \rangle \langle a^\dagger_\nu a_\sigma \rangle - \langle a^\dagger_\mu a_\sigma \rangle \langle a^\dagger_\nu a_\lambda \rangle = \delta_{\mu\lambda} \delta_{\nu\sigma} - \delta_{\mu\sigma} \delta_{\nu\lambda}
% \end{equation}
% Now, plugging this into the expression for the two-electron integral
% \begin{equation}
%     \frac{1}{2} \sum_{\mu\lambda\nu\sigma} g_{\mu\lambda\nu\sigma} a^\dagger_\mu a^\dagger_\nu a_\sigma a_\lambda = \frac{1}{2} \sum_{\mu\lambda\nu\sigma} g_{\mu\lambda\nu\sigma} \left( \delta_{\mu\lambda} \delta_{\nu\sigma} - \delta_{\mu\sigma} \delta_{\nu\lambda} \right) = \frac{1}{2} \sum_{\mu\nu} g_{\mu\mu\nu\nu} - g_{\mu\nu\nu\mu}
% \end{equation}
% The first term is the Hartree term and the second term is the exchange term. In the uniform electron gas, we assume that the Hartree term vanishes because it exactly cancels out with the attraction of the electron to the positive background, so we are only left with the exchange term, arising from the antisymmetry we have mandated on the wave function. In the chemist's notation, the two-electron integral can be written as:
% \begin{equation}
%     g_{\mu\lambda\nu\sigma} = [\mu\lambda|\nu\sigma] = \int d\mathbf{x}_1 d\mathbf{x}_2 \phi^*_\mu(\mathbf{x}_1) \phi_\lambda(\mathbf{x}_1) \frac{1}{|\mathbf{x}_1 - \mathbf{x}_2|} \phi^*_\nu(\mathbf{x}_2) \phi_\sigma(\mathbf{x}_2)
% \end{equation}
% Neglecting the spin for now, since we know what the final factor will be for the exchange term, we continue by just considering the position $\mathbf{r}$.
% The matrix element can be simplified to
% \begin{equation}
% \begin{aligned}
%     g_{\mu\lambda\nu\sigma} &=\frac{1}{\Omega } \int_{\Omega} d\mathbf{r}_1 d\mathbf{r}_2 e^{-i\mathbf{k}_\mu\cdot\mathbf{r}_1} e^{i\mathbf{k}_\lambda\cdot\mathbf{r}_1} \frac{1}{|\mathbf{r}_1 - \mathbf{r}_2|} e^{-i\mathbf{k}_\nu\cdot\mathbf{r}_2} e^{i\mathbf{k}_\sigma\cdot\mathbf{r}_2} \\
% \end{aligned}
% \end{equation}
% We want to simplify the term $\frac{1}{|\mathbf{r}_1 - \mathbf{r}_2|}$ in the expression for $g_{\mu\lambda\nu\sigma}$.
% Let's call the difference between the position vectors $\mathbf{r}_1$ and $\mathbf{r}_2$ as $\mathbf{r} = \mathbf{r}_1 - \mathbf{r}_2$. Then, the Fourier transform of the $f(\mathbf{r}) = 1/|\mathbf{r}|$ is given by:
% \begin{equation}
%     \tilde{f}(\mathbf{q})
% = \int d\mathbf{r} e^{-i \mathbf{q}\cdot\mathbf{r}} f(\mathbf{r})
% \end{equation}
% A standard integral table will show that the Fourier coefficient of $1/|\mathbf{r}|$ is:
% \begin{equation}
%     \tilde{f}(\mathbf{q}) = \frac{4\pi}{|\mathbf{q}|^2}
% \end{equation}
% Taking the inverse Fourier transform of this, we get:
% \begin{equation}
%     f(\mathbf{r}) = \frac{1}{|\mathbf{r}|} = \frac{4\pi}{\Omega } \int d\mathbf{q} \frac{e^{i\mathbf{q}\cdot\mathbf{r}}}{|\mathbf{q}|^2} \rightarrow  1/|\mathbf{r}_1 - \mathbf{r}_2| = \frac{4\pi}{\Omega } \int d\mathbf{q} \frac{e^{i\mathbf{q}\cdot\left(\mathbf{r}_1 - \mathbf{r}_2\right)}}{|\vb{q}|^2}
% \end{equation}
% Plugging this into the expression for $g_{\mu\lambda\nu\sigma}$ we get:
% \begin{equation}
% \begin{aligned}
%     g_{\mu\lambda\nu\sigma} &= \frac{4\pi}{\Omega ^{3}} \int d\mathbf{r}_1 d\mathbf{r}_2 e^{i(\mathbf{k}_\lambda - \mathbf{k}_\mu)\cdot\mathbf{r}_1} e^{i(\mathbf{k}_\sigma - \mathbf{k}_\nu)\cdot\mathbf{r}_2} \int d\mathbf{q} \frac{e^{i\mathbf{q}\cdot\left(\mathbf{r}_1 - \mathbf{r}_2\right)}}{|\vb{q}|^2} \\
% \end{aligned}
% \end{equation}
% Moving the integral over the wave vector and front and separating out the integrals over position vectors, we get:
% \begin{equation}
% \begin{aligned}
%     g_{\mu\lambda\nu\sigma} &= \frac{4\pi}{\Omega ^{3}} \int d\mathbf{q} \frac{1}{|\vb{q}|^2} \int d\mathbf{r}_1 e^{i(\mathbf{k}_\lambda - \mathbf{k}_\mu + \mathbf{q})\cdot\mathbf{r}_1} \int d\mathbf{r}_2 e^{i(\mathbf{k}_\sigma - \mathbf{k}_\nu - \mathbf{q})\cdot\mathbf{r}_2} \\
%     &= \frac{4\pi}{\Omega ^{3}} \int d\mathbf{q} \frac{1}{|\vb{q}|^2} \Omega  \delta(\mathbf{k}_\lambda - \mathbf{k}_\mu + \mathbf{q}) \Omega  \delta(\mathbf{k}_\sigma - \mathbf{k}_\nu - \mathbf{q}) \\
% \end{aligned}
% \end{equation}
% Canceling out the constant factors of $\Omega$:
% \begin{equation}
%     g_{\mu\lambda\nu\sigma} = \frac{4\pi}{\Omega} \int d\mathbf{q} \frac{1}{|\mathbf{q}|^2} \delta(\mathbf{k}_\lambda - \mathbf{k}_\mu + \mathbf{q}) \delta(\mathbf{k}_\sigma - \mathbf{k}_\nu - \mathbf{q})
% \end{equation}
% This tells us that the constraint $\mathbf{q} = \mathbf{k}_\mu - \mathbf{k}_\lambda = \mathbf{k}_\sigma - \mathbf{k}_\nu$ must be satisfied. We simplify the expression for $g_{\mu\lambda\nu\sigma}$ to:
% \begin{equation}
%     g_{\mu\lambda\nu\sigma} = \frac{4\pi}{\Omega} \frac{1}{|\mathbf{k}_\mu - \mathbf{k}_\lambda|^2} \delta_{\mathbf{k}_\mu - \mathbf{k}_\lambda, \mathbf{k}_\sigma - \mathbf{k}_\nu}
% \end{equation}
Now, we want to derive the expressions for the ERI operators in the 3D case. We know that the hearty operator in hearty folk theory in first quantization is defined by:
\begin{equation}
    J^{}_n(\mathbf{r}) = \int d\mathbf{r}^{\prime} \psi_n^{*}(\mathbf{r}^{\prime}) \frac{1}{\left|\mathbf{r} - \mathbf{r}^{\prime}\right|} \psi_n(\mathbf{r}^{\prime})
\label{eq:hartree}
\end{equation}
So in total it is given by:
\begin{equation}
    J(\mathbf{r}) = \sum_{N_{occ}} J^{}_n(\mathbf{r})
\end{equation}
In the plain wave bases, its matrix a limit is given by
\begin{equation}
    J_{\mu\nu} = \int_{\Omega } d\mathbf{r} \phi^*_\mu(\mathbf{r}) \left( \sum_{N_{occ}} \int d\mathbf{r}^{\prime} \psi_n^{*}(\mathbf{r}^{\prime}) \frac{1}{\left|\mathbf{r} - \mathbf{r}^{\prime}\right|} \psi_n(\mathbf{r} ^{\prime}) \right) \phi_\nu(\mathbf{r})
\end{equation}
Plugging in equation \ref{eq:hartree} we get:
\begin{equation}
    J_{\mu\nu} = \sum_{N_{occ}} \int_{\Omega } d\mathbf{r} \int d\mathbf{r}^{\prime} \phi^*_\mu(\mathbf{r}) \psi_n^{*}(\mathbf{r}^{\prime}) \frac{1}{\left|\mathbf{r} - \mathbf{r}^{\prime}\right|} \psi_n(\mathbf{r}^{\prime}) \phi_\nu(\mathbf{r})
\end{equation}
We know that the molecular orbitals $\psi_n^{\sigma }(\mathbf{r})$ are given by a linear combination of the plane waves as $\psi_n^{}(\mathbf{r}) = \sum_i c^{}_{n,i} \phi_i(\mathbf{r})$. Plugging this into the expression for $J_{\mu\nu}$ we get:
\begin{equation}
    J_{\mu\nu} = \sum_{N_{occ}} \int_{\Omega } d\mathbf{r} \int d\mathbf{r}^{\prime} \phi^*_\mu(\mathbf{r}) \left( \sum_{\lambda} c_{n,\lambda}^{*} \phi_\lambda(\mathbf{r}^{\prime}) \right) \frac{1}{\left|\mathbf{r} - \mathbf{r}^{\prime}\right|} \left( \sum_{\sigma } c_{n,\sigma } \phi_\sigma(\mathbf{r}^{\prime}) \right) \phi_\nu(\mathbf{r})
\end{equation}
Bringing the two sums over atomic orbitals and their coefficients out to the front, we get:
\begin{equation}
    J_{\mu\nu} = \sum_{\lambda\sigma} \left( \sum_{N_{occ}} \left( c_{n,\lambda}^{*} c_{n,\sigma } \right) \right) \int_{\Omega } d\mathbf{r} \int d\mathbf{r}^{\prime} \phi^*_\mu(\mathbf{r}) \phi_\lambda(\mathbf{r}^{\prime}) \frac{1}{\left|\mathbf{r} - \mathbf{r}^{\prime}\right|} \phi_\nu(\mathbf{r}) \phi_\sigma (\mathbf{r}^{\prime})
\end{equation}
In the first stricter formulation, we simplify to
\begin{equation}
    J_{\mu\nu} = \sum_{\lambda\sigma} P_{\lambda\sigma } \int_{\Omega } d\mathbf{r} \int d\mathbf{r}^{\prime} \phi^*_\mu(\mathbf{r}) \phi_\lambda(\mathbf{r}^{\prime}) \frac{1}{\left|\mathbf{r} - \mathbf{r}^{\prime}\right|} \phi_\nu(\mathbf{r}) \phi_\sigma (\mathbf{r}^{\prime})
\end{equation}
Defining the ERI as
\begin{equation}
    (\mu\nu|\lambda\sigma) = \int_{\Omega } d\mathbf{r} \int d\mathbf{r}^{\prime} \phi^*_\mu(\mathbf{r}) \phi^*_{\lambda }(\mathbf{r}^{\prime}) \frac{1}{\left|\mathbf{r} - \mathbf{r}^{\prime}\right|} \phi_{\nu}(\mathbf{r}) \phi_\sigma (\mathbf{r}^{\prime})
\end{equation}
We know this simplifies to
\begin{equation}
    (\mu\nu|\lambda \sigma) = \frac{4\pi}{\Omega} \int d\mathbf{q} \frac{1}{|\mathbf{q}|^2} \delta_{\mathbf{q}, \mathbf{G}_\mu - \mathbf{G}_\nu} \delta_{\mathbf{q}, \mathbf{G}_\lambda - \mathbf{G}_\sigma}
\end{equation}
With the delta functions this integral can be simplified to
\begin{equation}
    (\mu\nu|\lambda \sigma) = \frac{4\pi}{\Omega} \frac{1}{|\mathbf{G}_\mu - \mathbf{G}_\nu|^2} \delta_{\mathbf{G}_\mu - \mathbf{G}_\nu, \mathbf{G}_\lambda - \mathbf{G}_\sigma}
\end{equation}
Now, for the exchange term, now the spin is relevant and we fast forward to
\begin{equation}
    K^{\sigma }_{\mu\nu} = \sum_{\lambda \sigma } P_{\lambda \sigma }^{\sigma } (\mu\sigma | \lambda\nu)
\end{equation}
The integral is now
\begin{equation}
    (\mu\sigma | \lambda\nu) = \frac{4\pi}{\Omega} \int d\mathbf{q} \frac{1}{|\mathbf{q}|^2} \delta_{\mathbf{q}, \mathbf{G}_\mu - \mathbf{G}_\sigma} \delta_{\mathbf{q}, \mathbf{G}_\lambda - \mathbf{G}_\nu} = \frac{4\pi}{\Omega} \frac{1}{|\mathbf{G}_\mu - \mathbf{G}_\sigma|^2} \delta_{\mathbf{G}_\mu - \mathbf{G}_\sigma, \mathbf{G}_\lambda - \mathbf{G}_\nu}
\end{equation}
and we have the following expression for the exchange term:
\begin{equation}
    K^{\sigma }_{\mu\nu} = \sum_{\lambda} P_{\lambda \sigma }^{\sigma } (\mu\sigma | \lambda\nu) = \frac{4\pi}{\Omega} \sum_{\lambda} P_{\lambda \sigma }^{\sigma } \frac{1}{|\mathbf{G}_\mu - \mathbf{G}_\sigma|^2} \delta_{\mathbf{G}_\mu - \mathbf{G}_\sigma, \mathbf{G}_\lambda - \mathbf{G}_\nu}
\end{equation}  
% We have the following expression for the Hartree term:
% \begin{equation}
%     J^{\sigma }_{\mu\nu} = \sum_{ij} P_{ij}^{\sigma } (\mu\nu|ij)
% \end{equation}
% This leads to
% \begin{equation}
% J_{\vec{k}, p q}^\sigma=\frac{4 \pi}{\Omega} \sum_{\vec{k}^{\prime}} \sum_{i j} P_{\vec{k}^{\prime}, j i}^\sigma \frac{1}{\left|\vec{G}_j-\vec{G}_i\right|^2} \delta\left[\left(\vec{G}_j+\vec{G}_q\right)-\left(\vec{G}_i+\vec{G}_p\right)\right]
% \end{equation}
% We know the delta function insists $\vec{G}_j-\vec{G}_i = \vec{G}_p - \vec{G}_q$ which means that $\vec{G}_i = \vec{G}_j - \vec{G}_p + \vec{G}_q$. Substituting this in gifts
% \begin{equation}
% J_{\vec{k}, p q}^\sigma=\frac{4 \pi}{\Omega} \sum_{\vec{k}^{\prime}} \sum_{i j} P_{\vec{k}^{\prime}, j, j - (p-q)}^\sigma \frac{1}{\left|\vec{G}_j-\left( \vec{G}_j - \vec{G}_p + \vec{G}_q \right)\right|^2}
% \end{equation}
% Defining $Q = \vec{G}_p - \vec{G}_q$ and $P_{j, j-Q}^\sigma = P_{\vec{k}^{\prime}, j, j - (p-q)}^\sigma$ we get
% \begin{equation}
% J_{\vec{k}, p q}^\sigma=\frac{4 \pi}{|Q|^2\Omega } \sum_{\vec{k}^{\prime}} \sum_{j} P_{j, j-Q}^\sigma
% \end{equation}
% This can be rewritten as
% \begin{equation}
% \begin{aligned}
% K_{pq}^\sigma & = \frac{4 \pi}{\Omega} \sum_Q P_{p-Q, q-Q}^{\sigma \sigma^{\prime}} \frac{1}{\left|\mathbf{Q}\right|^2} .
% \end{aligned}
% \end{equation}
% We have to keep in mind that $Q$ is defined as the vector of all possible momentum transfers. Consider that we have $max_n = 1$, but we can only turned out to [1,1,0], which has ke of $1^2+1^2 = 2$.
% . If I simply say that the momentum transfer vectors can range from [-2,-2,-2] to [2,2,2], this is wasteful because the maximum I need to consider is [2,2,0], with ke $(2nmax)^2 + (2nmax)^2$. or pw = [1,3,2] naive would have [6,6,6] possible, but we only need to up to [2,6,4] or could be [4,4,3]
% So one could imagine a situation where $p = [3,3,3]$ and then even $Q = [6,6,6]$ would be a valid momentum transfer as this would generate a plane wave with $p-Q = [3,3,3] - [6,6,6] = [-3,-3,-3]$, which is still valid. Let's think about this more formally. It isn't enough to just concerto that $Q=2p$ because a $Q$ of $[5,5,5]$ is valved but it is not any multiple of $p$. This means that we need to consider additions of to $2p$ and not just multiply $p$ by an integer. $Q$ could also be something that is not just a list of the same indenture; we could have $p = [2,0,1]$ and $Q = [1,-1,3$ which would give the difference $p-Q = [1,1,-2]$ which that could still be a valid plane wave. 
\subsubsection{Alternative derivation}
We can derive the two-electron term in another way. We know this is given by
\begin{equation}
    \hat{V}_{ee} = \frac{1}{2 \Omega } \sum_{G_1 G_2 G_3 G_4} \frac{4\pi}{|G_2-G_3|^2} a^\dagger_{G_1} a^\dagger_{G_2} a_{G_3} a_{G_4}
\end{equation}
We know that the total momentum transfer must be conserved, so $G_1 + G_2 = G_3 + G_4$, and $G_4 = G_1 + G_2 - G_3$. Defining $Q=G_2-G_3$, we know that $G_4 = G_1 + Q$, whereas then $G_3 = G_2 - Q$. Redefining $G_1 \equiv G$ and $G_2 \equiv G^\prime$, we can write the two-electron term as:
\begin{equation}
    \hat{V}_{ee} = \frac{1}{2 \Omega } \sum_{G G^\prime Q} \frac{4\pi}{|Q|^2} a^\dagger_{G} a^\dagger_{G^\prime} a_{G^\prime - Q} a_{G + Q}
\end{equation}
Applying Wick's theorem gives:
\begin{equation}
    \hat{V}_{ee} = \frac{1}{2 \Omega } \sum_{G G^\prime Q} \frac{4\pi}{|Q|^2} \left( P_{G,G+Q}P_{G^\prime, G^\prime - Q} - P_{G, G^\prime - Q}P_{G^\prime, G + Q} \right),
\end{equation}
where we use the usual definition that $\expval{a^\dagger_{G_1} a_{G_2}} = P_{G_1, G_2}$, which is the density matrix.
We are just interested in the second term for exchange, which is 
\begin{equation}
    \Sigma _{x} = -\frac{1}{2 \Omega } \sum_{G G^\prime} \Lambda _{G G^\prime}
\end{equation}
where we defined
\begin{equation}
    \Lambda _{G G^\prime} = \sum_{Q} \frac{4\pi}{|Q|^2} P_{G, G^\prime - Q}P_{G^\prime, G + Q}
\end{equation}
Here we might have the similar thing with the Fourier transform of $1/|\mathbf{r}|$ being $\frac{2\pi}{\mathbf{q}}$ and the two-electron integral simplifying to:
\begin{equation}
    g_{\mu\lambda\nu\sigma} = \frac{2\pi}{A} \frac{1}{|\mathbf{k}_\mu - \mathbf{k}_\lambda|} \delta_{\mathbf{k}_\mu - \mathbf{k}_\lambda, \mathbf{k}_\sigma - \mathbf{k}_\nu}
\end{equation}


\section{Kinetic energy}

We started by specifying a kinetic energy cutoff. In atomic units, the kinetic energy of a plane wave is given by:
\begin{equation},
    E_{\text{kinetic}} = \frac{\vb{k}^2}{2}
\end{equation}
Now if we try to evaluate $\vb{k}^2$ for a plane wave with wave vector $\vb{k} = (k_x, k_y, k_z)$, we get:
\begin{equation}
    k^2 = k_x^2 + k_y^2 + k_z^2 =  \left(\frac{2\pi}{L}\right)^2 \left(n_x^2 + n_y^2 + n_z^2\right)
\end{equation}
where \( L \) is the length of the box and \( n_x, n_y, n_z \) are integers. In terms of the Wigner-Seitz radius \( r_s \), the volume of the cell is given by the volume of the sphere of a single electron multiplied by the number of electrons
\begin{equation}
    V = \left(\frac{4\pi N}{3}\right) r_s^3
\end{equation}
\emph{We can approximate it as a box with the same volume} and \( V= L^3 \). So, we get an expression for \( L \) as:
\begin{equation}
    L = \left( \frac{4\pi N}{3} \right)^{1/3} r_s
\end{equation}
Let $C = \frac{1}{2}\left(\frac{2 \pi}{L}\right)^2$.
Plugging this into the expression for $E_{\text{kinetic}}$ we get:
\begin{equation}
    E_{\text{kinetic}} = C \left(n_x^2 + n_y^2 + n_z^2\right) = C \left(n_x^2\right) + C \left(n_y^2 + n_z^2\right)
\end{equation}
Therefore, we know that the kinetic anergy cut off should scale with as $N^{-2/3}$ and $r_s^{-2}$.
\subsection{2D case}
As before the kinetic energy of a plain wave is given by the same thing:
\begin{equation}
    E_{\text{kinetic}} = \frac{\vb{k}^2}{2}
\end{equation}
Now if we try to evaluate $\vb{k}^2$ for a plane wave with wave vector $\vb{k} = (k_x, k_y)$, we get:
\begin{equation}
    k^2 = k_x^2 + k_y^2 =  \left(\frac{2\pi}{L}\right)^2 \left(n_x^2 + n_y^2\right)
\end{equation}
where $L$ is the length of the box and $n_x, n_y$ are integers. In terms of the Wigner-Seitz radius $r_s$, the area of the cell is given by the area of the circle of a single electron multiplied by the number of electrons
\begin{equation}
    A = \pi N r_s^2
\end{equation}
\emph{We can approximate it as a box with the same area} and $A= L^2$. So, we get an expression for $L$ as:
\begin{equation}
    L = \sqrt{\pi N} r_s
\end{equation}
Plugging this into the expression for $E_{\text{kinetic}}$ we get:
\begin{equation}
    E_{\text{kinetic}} = \frac{1}{2}\left(\frac{2 \pi}{\sqrt{\pi N} r_s}\right)^2 \left(n_x^2 + n_y^2\right) = 2\pi N^{-1} r_s^{-2} \left(n_x^2 + n_y^2\right)
\end{equation}
\section{SCF procedure}
The Fermi energy is defined as the energy at which the probability of finding an electron is 1/2. In the uniform electron gas, the Fermi energy is the energy of the HOMO and is given in natural units as:
\begin{equation}
    E_F = \frac{\left(\frac{9 \pi^4}{16}\right)^{2 / 3}}{r_s^2}
\end{equation}
Once we specify the kinetic energy cutoff, we are left with a number of valid plane wave basis states $N_{\text{PW}}$. We construct the kinetic and Coulomb matrices using this basis. For the initial guess of the one-electron density matrix in the restricted formalism, we just have its diagonal filled with 2s up to $N_\text{elec}/2$ and 0s for the rest.
\begin{equation}
    P^{(0)}_{\mu\nu} = \begin{cases}
    2 & \text{if } \mu = \nu \leq N_\text{elec}/2 \\
    0 & \text{otherwise}
    \end{cases}
\end{equation}
Next, we construct a Fock matrix with:
\begin{equation}
    F_{\mu\nu} = H_{\mu\nu}^{\text{core}} + \sum_{\lambda\sigma} P_{\lambda\sigma} \left(g_{\mu\lambda\nu\sigma} - \frac{1}{2}g_{\mu\nu\sigma\lambda}\right)
\end{equation}
and then diagonalize it to get the new orbital coefficients $C_{\mu i}$ and the single particle energies $\varepsilon_i$:
\begin{equation}
    \varepsilon_i = h_{ii} + \sum_{a}^{N_\text{elec}/2} \left(2J_{ia} - K_{ia}\right)
\end{equation}
 We also construct a new density matrix using the new orbital coefficients:
\begin{equation}
    P_{\mu\nu} = 2\sum_{i=1}^{N_\text{elec}/2} C_{\mu i} C_{\nu i}
\end{equation}
 The convergence criteria are for the change in energy and the norm of the density matrix between iterations to be minimal.
The formular for the restricted Hartree-Fock energy is not just the sum of the single particle energies, because this double counts the electron-electron repulsion. The correct formula is:
\begin{equation}
    E_{HF} = \frac{1}{2} \sum_{i=1}^{N_\text{elec}/2} \left(\varepsilon_i + h_{ii}\right)
\end{equation}
Then there is the Madeleung constant two take into account that just provides a correction to the energy:
\begin{equation}
E_M \approx-2.837297 \times\left(\frac{3}{4 \pi}\right)^{1 / 3} N^{2 / 3} r_\pi^{-1}
\end{equation}
By plugging in the single protocol energy, we see that this simplifies to:
\begin{equation}
    E_{HF} = \frac{1}{2} \sum_{i=1}^{N_\text{elec}/2} \left(h_{ii} + \sum_{a}^{N_\text{elec}/2} \left(2J_{ia} - K_{ia}\right) + h_{ii}\right) = \sum_{i=1}^{N_\text{elec}/2} h_{ii} + \frac{1}{2} \sum_{i=1}^{N_\text{elec}/2} \sum_{a}^{N_\text{elec}/2} \left(2J_{ia} - K_{ia}\right)
\end{equation}
which is NOT the same as:
\begin{equation}
    \sum_{i=1}^{N_\text{elec}/2} \varepsilon_i = \sum_{i=1}^{N_\text{elec}/2} h_{ii} + \sum_{i=1}^{N_\text{elec}/2} \sum_{a}^{N_\text{elec}/2} \left(2J_{ia} - K_{ia}\right) 
\end{equation}
we can also equivalently write the energy as:
\begin{equation}
    E_0=\frac{1}{2} \sum_\mu \sum_v P_{v u}\left(H_{\mu \nu}^{\mathrm{core}}+F_{\mu \nu}\right)
\end{equation}
In order to determine the spin polarization, we know that $N_{\alpha }+N_{\beta } = N_{\text{elec}}$ and $\rho = \frac{N_{\alpha }-N_{\beta }}{N_{\text{elec}}}$, where $N_{\alpha }$ and $N_{\beta }$ are the number of alpha and beta electrons respectively with spin polarization $\rho$. Solving these coupled equations gives
\begin{equation}
    N_{\alpha } = \frac{N_{\text{elec}}}{2} \left(1 + \rho\right) \quad N_{\beta } = \frac{N_{\text{elec}}}{2} \left(1 - \rho\right)
\end{equation}
\end{document}