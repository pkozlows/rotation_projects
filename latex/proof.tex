\documentclass[12pt]{article}
\usepackage{amsmath}
\usepackage{amssymb}
\usepackage{graphicx}
\usepackage{physics}
\usepackage{hyperref}

\setlength{\oddsidemargin}{0in}
\setlength{\evensidemargin}{0in}
\setlength{\topmargin}{0in}
\setlength{\textwidth}{6.5in}
\setlength{\textheight}{8.5in}

\title{Proving AC}
\author{Patryk Kozlowski}
\date{\today}
\begin{document}
\maketitle
\section{Slow}
We want to solve for the self-energy whose form along the real axis is:
\begin{equation}
\Sigma\left(\mathbf{r}, \mathbf{r}^{\prime}, \omega\right)=\frac{i}{2 \pi} \int_{-\infty}^{\infty} e^{i\omega ^{\prime}\eta }d \omega^{\prime} G_{0}\left(\mathbf{r}, \mathbf{r}^{\prime}, \omega+\omega^{\prime}\right) W_{0}\left(\mathbf{r}, \mathbf{r}^{\prime}, \omega^{\prime}\right)
\end{equation}
We want to evaluate the matrix element of the self-energy between two molecular orbitals. Note that in this process, the crystal momentum must be conserved. Additionally, we apply the resolution of the identity operator:
\begin{equation}
    \bra{n\mathbf{k}} \Sigma (\omega) \ket{n^\prime\mathbf{k}} = \frac{i}{2 \pi} \sum_{m\mathbf{k}^{\prime}} \int_{-\infty}^{\infty} e^{i\omega ^{\prime}\eta }d \omega^{\prime} \bra{n\mathbf{k}} G_0(\omega + \omega^{\prime})\ket{m\mathbf{k}^{\prime}} \bra{m\mathbf{k}^{\prime}}W_0( \omega^{\prime}) \ket{n^\prime\mathbf{k}}
\end{equation}
Now, we know that the Lehmann representation of the noninteracting Green's function is:
\begin{equation}
G_0\left(\mathbf{r}, \mathbf{r}^{\prime}, \omega\right)=\sum_{o \mathbf{q} } \frac{\psi_{o \mathbf{q}}(\mathbf{r}) \psi_{o \mathbf{q}}^{*}\left(\mathbf{r}^{\prime}\right)}{\omega-\epsilon_{o \mathbf{q}}+i \eta \operatorname{sgn}\left(\epsilon_{o \mathbf{q}}-\mu\right)}
\end{equation}
where $\mu$ is the chemical potential. Plugging this into the above equation gives:
\begin{equation}
    \boldsymbol{\Sigma}_{n n^{\prime}}(\mathbf{k}, \omega) = \frac{i}{2 \pi} \sum_{m\mathbf{k}^\prime o\mathbf{q}} \int_{-\infty}^{\infty} d \omega^{\prime} e^{i\omega ^{\prime}\eta }  \frac{\bra{n\mathbf{k}o\mathbf{q}}\ket{o\mathbf{q}m\mathbf{k}^\prime}}{(\omega + \omega^{\prime}) -\epsilon_{o \mathbf{q}} + i \eta \operatorname{sgn}\left(\epsilon_{o \mathbf{q}}-\mu\right)} \times \bra{m\mathbf{k}^\prime}W_0( \omega^{\prime}) \ket{n^\prime\mathbf{k}}
\end{equation}
We know that for the integrand to be nonzero, the crystal momentum must be conserved, so that \(\mathbf{k} + \mathbf{q} = \mathbf{k}^{\prime} + \mathbf{q} \implies \mathbf{k} = \mathbf{k}^{\prime}\) and also by orthogonality of the basis functions we see that \(m\equiv n\), so that the integrand becomes:
\begin{equation}
    \boldsymbol{\Sigma}_{n n^{\prime}}(\mathbf{k}, \omega) = \frac{i}{2 \pi} \sum_{o\mathbf{q}} \int_{-\infty}^{\infty} d \omega^{\prime} e^{i\omega ^{\prime}\eta } \frac{\bra{n\mathbf{k}}W_0(\omega^{\prime}) \ket{n^{\prime}\mathbf{k}}}{(\omega + \omega^{\prime}) - \epsilon_{o \mathbf{q}} + i \eta \operatorname{sgn}\left(\epsilon_{o \mathbf{q}}-\mu\right)}
\end{equation}
So the Green's function will bring poles at \(\omega = \epsilon_{0 \mathbf{q}} - \omega' + i\eta\operatorname{sgn}(\mu - \epsilon_{0 \mathbf{q}})\). Now, we know that the screened Coulomb interaction has the expansion in terms of the bare Coulomb potential \(v\) and the density response function \(\chi_0\) as \(W_0 = v + v\chi_0 v + v\chi_0 v\chi_0 v + \cdots = v\left(1 + \chi_0 v + \chi_0 v \chi_0 v + \cdots\right) = v\left(1 - \chi_0 v\right)^{-1}\), where we recognize the dielectric function as \(\epsilon_0 = 1 - \chi_0 v\) so we can express the screened Coulomb interaction as
\begin{equation}
   W_0(\mathbf{r}, \mathbf{r}^{\prime}, \omega) = \frac{v(\mathbf{r}, \mathbf{r}^{\prime})}{1 - \left(\chi_0v\right)(\mathbf{r}, \mathbf{r}^{\prime}, \omega)}
\end{equation}
recalling that the bare Coulomb interaction should be independent of frequency. So in our quest to find poles of $W_0$, we are really just looking for poles of the $\chi_0$. $\chi_0$ is given by:
\begin{equation}
\chi_{0}\left(\mathbf{r}, \mathbf{r}^{\prime}, \omega\right)=\sum_{r\mathbf{k} s \mathbf{k}^{\prime}}\left(f_{r \mathbf{k}}-f_{s \mathbf{k}^{\prime}}\right) \frac{\psi_{r \mathbf{k}}(\mathbf{r}) \psi_{r \mathbf{k}}^{*}\left(\mathbf{r}^{\prime}\right) \psi_{s \mathbf{k}^{\prime}}\left(\mathbf{r}^{\prime}\right) \psi_{s \mathbf{k}^{\prime}}^{*}(\mathbf{r})}{\omega-\left(\epsilon_{r \mathbf{k}}-\epsilon_{s \mathbf{k}^{\prime}}\right)+i \eta \operatorname{sgn}\left(\epsilon_{r \mathbf{k}}-\epsilon_{s \mathbf{k}^{\prime} } - \mu\right)}
\end{equation}
where the occupations of the KS states \(r\mathbf{k}(s\mathbf{k}^{\prime})\) with energies \(\epsilon_{r\mathbf{k}}(\epsilon_{s\mathbf{k}^{\prime}})\) are given by the Fermi-Dirac distribution \(f_{r\mathbf{k}}(f_{s\mathbf{k}^{\prime}})\), which is just a step function at zero temperature. Notice that the occupation factor will always be 0 unless \(rs\) form an occupied-virtual pair. So we can separate the density response into two terms, one where \(\delta_{ri}\) and \(\delta_{sa}\) and the other with \(\delta_{ra}\) and \(\delta_{si}\), where \(i\) and \(a\) are occupied and virtual indices, respectively. This allows us to now combine with the bare Coulomb potential in order to form the polarizability $\Pi\equiv \chi_0v$ as:
\begin{equation}
\Pi\left(\mathbf{r}, \mathbf{r}^{\prime}, \omega\right)=\sum_{i\mathbf{k}a\mathbf{k}^{\prime}}\frac{\psi_{i\mathbf{k} }(\mathbf{r}) \psi_{i\mathbf{k}}^{*}\left(\mathbf{r}^{\prime}\right)\frac{1}{|\mathbf{r}-\mathbf{r}^\prime|} \psi_{a\mathbf{k}^{\prime}}\left(\mathbf{r}^{\prime}\right) \psi_{a\mathbf{k}^{\prime}}^{*}(\mathbf{r})}{\omega+\left(\Omega_{i\mathbf{k}a\mathbf{k}^{\prime}}\right)-i\eta } - \sum_{a\mathbf{k}i\mathbf{k}^{\prime}}\frac{\psi_{a\mathbf{k}}(\mathbf{r}) \psi_{a\mathbf{k}}^{*}\left(\mathbf{r}^{\prime}\right) \frac{1}{|\mathbf{r}-\mathbf{r}^\prime|}\psi_{i\mathbf{k}^{\prime}}\left(\mathbf{r}^{\prime}\right) \psi_{i\mathbf{k}^{\prime}}^{*}(\mathbf{r})}{\omega-\left(\Omega_{i\mathbf{k}a\mathbf{k}^{\prime}}\right)+i\eta },
\end{equation}
where we define the KS eigenvalue differences as \(\Omega_{i\mathbf{k}a\mathbf{k}^{\prime}} = \epsilon_{a\mathbf{k}} - \epsilon_{i\mathbf{k}^{\prime}}\). Note that this expression implies that the poles of the polarizability are in the negative complex plane for positive $\Omega $ and vice versa for negative $\Omega $. So sandwiching this operator in between the molecular or brutal bases gives:
\begin{equation}
    \bra{n\mathbf{k}} \Pi (\omega) \ket{n^\prime\mathbf{k}} = \sum_{iajb \mathbf{k} \mathbf{k}^{\prime}} \frac{(ia\mid jb)}{(\omega + \Omega_{ia\mathbf{k}j\mathbf{k}^{\prime}})-i\eta} - \sum_{aibj \mathbf{k} \mathbf{k}^{\prime}} \frac{(ai\mid bj)}{(\omega - \Omega_{ia\mathbf{k}j\mathbf{k}^{\prime}})+i\eta}
\end{equation}
But we will proceed with an RPA calculation anyways in order to solve for the excitation energies and their corresponding eigenvectors. So we see that we can get the poles of the screened Coulomb interaction by the poles of the polarizability, which are $\omega = \Omega_{ia\mathbf{k}j\mathbf{k}^{\prime}} - i\eta$ and $\omega = \Omega_{ia\mathbf{k}j\mathbf{k}^{\prime}} + i\eta$, suggesting that they are in the upper complex plane for positive $\Omega_{ia\mathbf{k}j\mathbf{k}^{\prime}}$ and vice versa for negative $\Omega_{ia\mathbf{k}j\mathbf{k}^{\prime}}$. This suggests that the notation in the $G_0W_0$ literature is confusing because they always say that to solve for the $\chi_0$ in the RPA, but if we are actually dealing with $\chi_0$, which is the Kohn-Sham density response function, then we don't use the RPA, where the density response function is solved for using a Dyson-like equation:
\begin{equation}\label{eq:dyson}
    \chi^{\lambda}\left(\mathbf{r}, \mathbf{r}^{\prime}, i \omega\right) = \chi^{0}\left(\mathbf{r}, \mathbf{r}^{\prime}, i \omega\right) 
    + \int d \mathbf{r}_{1} d \mathbf{r}_{2} \chi^{0}\left(\mathbf{r}, \mathbf{r}_{1}, i \omega\right)\left[\frac{\lambda}{\left|\mathbf{r}_{1}-\mathbf{r}_{2}\right|}+f_{\mathrm{xc}}^{\lambda}\left(\mathbf{r}_{1}, \mathbf{r}_{2}, i \omega\right)\right] \chi^{\lambda}\left(\mathbf{r}_{2}, \mathbf{r}^{\prime}, \omega\right)
\end{equation}
where the parameter $\lambda$ controls the amount of interaction in the system, ranging from $\lambda = 0$ for the KS reference system to $\lambda = 1$ for the fully interacting system. The $f_{\mathrm{xc}}^{\lambda}$ is the exchange-correlation kernel, which is set to zero for the RPA. So it makes sense that the numerator of the expression for the screened Coulomb interaction should be given a construction of the ERIs with the excitation factors in a transition density defined as:
\begin{equation}
    w_{pq}^{\mu} = \sum_{ia} (pq|ia) \left(X_{ia}^{\mu} + Y_{ai}^{\mu}\right)
\end{equation}
where we have defined the excitation and de-excitation vectors at the excitation index $\mu$ as $X_{ia}^{\mu}$ and $Y_{ai}^{\mu}$, respectively.
I am not sure how to connect this with the known expression $v\epsilon ^{-1}$; I see the similarities given that we are contracting an ERI with what we get from the RPA calculation that is connected to the polarizability, but can't connect exactly.
We want to figure out how this matches with my proposed $O(N^6)$ expression, which was
\begin{equation}
    \Sigma_{pp}^{\text{corr}}(\omega) = \sum_{\mu }^{\text{RPA}}\left(\sum_{i}^{\text{occupied}} \frac{w_{pi}^{\mu }w_{ip}^{\mu }}{\omega -(\epsilon _{i}-\Omega  _{\mu })}+ \sum_{a}^{\text{virtual}} \frac{w_{pa}^{\mu }w_{ap}^{\mu }}{\omega -(\epsilon _{a}+\Omega  _{\mu })}\right)
\end{equation}
The expression I get adapted for k-points from GPT is:
\begin{equation}
\Sigma_{n n^\prime}^{\text {corr }}(\mathbf{k}, \omega)=\sum_{\mathbf{q}} \sum_\mu^{\text {RPA }}\left(\sum_i^{\text {occupied }} \frac{w_{n i}^\mu(\mathbf{k}, \mathbf{q}) w_{i n^\prime}^\mu(\mathbf{k}-\mathbf{q}, \mathbf{q})}{\omega-\left(\epsilon_{i, \mathbf{k}-\mathbf{q}}-\Omega_{\mu, \mathbf{q}}\right)}+\sum_a^{\text {virtual }} \frac{w_{n a}^\mu(\mathbf{k}, \mathbf{q}) w_{a n^\prime}^\mu(\mathbf{k}-\mathbf{q}, \mathbf{q})}{\omega-\left(\epsilon_{a, \mathbf{k}-\mathbf{q}}+\Omega_{\mu, \mathbf{q}}\right)}\right)
\end{equation}

\section{Setup}
We start with the original form for the self-energy along the real axis:
\begin{equation}
\Sigma\left(\mathbf{r}, \mathbf{r}^{\prime}, \omega\right)=\frac{i}{2 \pi} \int_{-\infty}^{\infty} d \omega^{\prime} e^{i \omega^{\prime} \eta} G_{0}\left(\mathbf{r}, \mathbf{r}^{\prime}, \omega+\omega^{\prime}\right) W_{0}\left(\mathbf{r}, \mathbf{r}^{\prime}, \omega^{\prime}\right)
\end{equation}
But to avoid the poles, we need to evaluate along the imaginary axis, so the problem becomes:
\begin{equation*}
\Sigma\left(\mathbf{r}, \mathbf{r}^{\prime}, i \omega\right)=-\frac{1}{2 \pi} \int_{-\infty}^{\infty} d \omega^{\prime} G_{0}\left(\mathbf{r}, \mathbf{r}^{\prime}, i \omega+i \omega^{\prime}\right) W_{0}\left(\mathbf{r}, \mathbf{r}^{\prime}, i \omega^{\prime}\right) \tag{16}
\end{equation*}
We are interested in evaluating the matrix elements of this operator in the molecular orbital basis. Note that both molecular orbitals must have the same crystal momentum in order for it to be conserved in this process. We also apply the identity operator:
\begin{equation}
    \bra{n\mathbf{k}} \Sigma (i \omega) \ket{n^\prime\mathbf{k}} = -\frac{1}{2 \pi} \sum_{m\mathbf{k}^\prime}\int_{-\infty}^{\infty} d \omega^{\prime} \bra{n\mathbf{k}} G_0(i \omega + i \omega^{\prime}) \ket{m\mathbf{k^\prime}}\bra{m\mathbf{k^\prime}}W_0(i \omega^{\prime}) \ket{n^\prime\mathbf{k}}
\end{equation}
The noninteracting Green's function has the form:
\begin{equation*}
G_{0}\left(\mathbf{r}, \mathbf{r}^{\prime}, i \omega\right)=\sum_{m \mathbf{k}_{m}} \frac{\psi_{m \mathbf{k}_{m}}(\mathbf{r}) \psi_{m \mathbf{k}_{m}}^{*}\left(\mathbf{r^\prime}\right)}{i \omega+\epsilon_{F}-\epsilon_{m \mathbf{k}_{m}}} \implies G_{0}\left(\mathbf{k} - \mathbf{q}, i \omega + i \omega^{\prime}\right) = \sum_{m \mathbf{k}-\mathbf{q}} \frac{\psi_{m \mathbf{k}-\mathbf{q}} \psi_{m \mathbf{k}-\mathbf{q}}^{*}}{i \left(\omega + \omega^{\prime}\right) + \epsilon_{F} - \epsilon_{m \mathbf{k}-\mathbf{q}}}
\end{equation*}
so that the above equation simplifies to:
\begin{equation}
    \boldsymbol{\Sigma}_{n n^{\prime}}(\mathbf{k}, i \omega) = -\frac{1}{2 \pi N_{\mathbf{k}}} \sum_{m \mathbf{q}} \int_{-\infty}^{\infty} d \omega^{\prime} \frac{(n\mathbf{k}, m\mathbf{k}-\mathbf{q} \mid W_0 (\mathbf{q}, i\omega )\mid m \mathbf{k}-\mathbf{q}, n^{\prime}\mathbf{k})}{i \left(\omega + \omega^{\prime}\right) + \epsilon_{F} - \epsilon_{m \mathbf{k}-\mathbf{q}}}
\end{equation}
\section{Screened Coulomb Interaction}
\begin{align*}
    \left(n\mathbf{k}, m\mathbf{k}-\mathbf{q}\left|W_{0}(\mathbf{q}, i\omega )\right| m\mathbf{k}-\mathbf{q}, n^{\prime}\mathbf{k}\right) &= \int \int d \mathbf{r}_1 d \mathbf{r}_2 \psi_{n\mathbf{k}}^{*}(\mathbf{r}_1) \psi_{m\mathbf{k}-\mathbf{q}}(\mathbf{r}_1) W_0(\mathbf{q}, \mathbf{r}_1, \mathbf{r}_2, i\omega ) \psi_{m\mathbf{k}-\mathbf{q}}^{*}(\mathbf{r}_2) \psi_{n^{\prime}\mathbf{k}}(\mathbf{r}_2) \\
\end{align*}
We expand the orbital pair product $\psi_{n \mathbf{k}}^{*}(\mathbf{r}) \psi_{m \mathbf{k}-\mathbf{q}}(\mathbf{r})$ in the auxiliary basis

\begin{equation*}
\psi_{n \mathbf{k}}^{*}(\mathbf{r}) \psi_{m \mathbf{k}-\mathbf{q}}(\mathbf{r})=\sum_{P} b_{P \mathbf{q}}^{n \mathbf{k}, m \mathbf{k}-\mathbf{q}} \phi_{P \mathbf{q}}(\mathbf{r}) \tag{20}
\end{equation*}
and
\begin{equation}
    \psi_{m\mathbf{k}-\mathbf{q}}^{*}(\mathbf{r}) \psi_{n^{\prime}\mathbf{k}}(\mathbf{r}) = \sum_{Q} b_{Q(-\mathbf{q})}^{m\mathbf{k}-\mathbf{q}, n^{\prime}\mathbf{k}} \phi_{Q(-\mathbf{q})}(\mathbf{r})
\end{equation}
where we have recognized the fact that in the former there is a momentum transfer of $\mathbf{q}$, and in the latter, there is a momentum transfer of $-\mathbf{q}$.
Substituting in gives
\begin{align}
    \left(n\mathbf{k}, m\mathbf{k}-\mathbf{q}\left|W_{0}(\mathbf{q}, i\omega )\right| m\mathbf{k}-\mathbf{q}, n^{\prime}\mathbf{k}\right)\\ = \sum_{PQ} b_{P\mathbf{q}}^{n\mathbf{k}, m\mathbf{k}-\mathbf{q}} \left[\iint d\mathbf{r}_1 d\mathbf{r}_2 \phi_{P\mathbf{q}}(\mathbf{r}_1) W_0(\mathbf{q}, \mathbf{r}_1, \mathbf{r}_2, i\omega ) \phi_{Q(-\mathbf{q})}(\mathbf{r}_2)\right] b_{Q(-\mathbf{q})}^{m\mathbf{k}-\mathbf{q}, n^{\prime}\mathbf{k}}
\end{align}
with

\begin{equation}
b_{P \mathbf{q}}^{n \mathbf{k}, m \mathbf{k}-\mathbf{q}}=\sum_{R}(n \mathbf{k}, m \mathbf{k}-\mathbf{q} \mid R \mathbf{q}) \cdot \mathbf{J}_{R P}^{-1}(\mathbf{q})
\label{eq:nonchol}
\end{equation}
Now is a good place to recall their definition of the density fitting where the ERIs are represented as:


\begin{equation*}
\left(p \mathbf{k}_{p} q \mathbf{k}_{q} \mid r \mathbf{k}_{r} s \mathbf{k}_{s}\right)=\sum_{P Q}\left(p \mathbf{k}_{p} q \mathbf{k}_{q} \mid P \mathbf{k}_{p q}\right) \mathbf{J}_{P Q}^{-1}\left(Q \mathbf{k}_{r s} \mid r \mathbf{k}_{r} s \mathbf{r}_{s}\right), \tag{11}
\end{equation*}
with
\begin{align*}
\mathbf{J}_{P Q}(\mathbf{k}) & =\iint d \mathbf{r} d \mathbf{r}^{\prime} \phi_{P(-\mathbf{k})}(\mathbf{r}) \frac{1}{\left|\mathbf{r}-\mathbf{r}^{\prime}\right|} \phi_{Q \mathbf{k}}\left(\mathbf{r}^{\prime}\right),  \tag{12}\\
\left(Q \mathbf{k}_{r s} \mid r \mathbf{k}_{r} s \mathbf{k}_{s}\right) & =\iint d \mathbf{r} d \mathbf{r}^{\prime} \phi_{Q \mathbf{k}_{r s}}(\mathbf{r}) \frac{1}{\left|\mathbf{r}-\mathbf{r}^{\prime}\right|} \phi_{r \mathbf{k}_{r}}^{*}\left(\mathbf{r}^{\prime}\right) \phi_{s \mathbf{k}_{s}}\left(\mathbf{r}^{\prime}\right) . \tag{13}
\end{align*}
Note that these $b$ are then not yet our Cholesky vectors, since each one contains $\frac{|\mathbf{r}-\mathbf{r}^{\prime}|}{|\mathbf{r}-\mathbf{r}^{\prime}|}$ scaling, i.e., there should be a factor of $\mathbf{J}^{-\frac{1}{2}}$  instead of $\mathbf{J}^{-1}$ in \ref{eq:nonchol} if we are to apply the Cholesky vectors.
At this point, we use the expansion of the screened Coulomb interaction:
\begin{align}
    W_0 &= v + v\chi_0 v + v\chi_0 v\chi_0 v + \cdots\\
    &= v(1 + \chi_0 v + \chi_0 v \chi_0 v + \cdots)\\
    &= v^{1/2} \left(1 - \chi_0\right)^{-1} v^{1/2}
\end{align}
simplifying to 
\begin{align}
    \left(n\mathbf{k}, m\mathbf{k}-\mathbf{q}\left|W_{0}(\mathbf{q}, i\omega )\right| m\mathbf{k}-\mathbf{q}, n^{\prime}\mathbf{k}\right) &= \sum_{PQ} b_{P\mathbf{q}}^{n\mathbf{k}, m\mathbf{k}-\mathbf{q}} \left[\mathbf{J^{\frac{1}{2}}}\left( \mathbf{I} - \boldsymbol{\Pi}(\mathbf{q}, i\omega ) \right) \mathbf{J^{\frac{1}{2}}}\right]^{-1}_{PQ} b_{Q(-\mathbf{q})}^{m\mathbf{k}-\mathbf{q}, n^{\prime}\mathbf{k}}\\
    &= \sum_{PQ} v_{P}^{n m}\left[\mathbf{I}-\mathbf{\Pi}\left(\mathbf{q}, i \omega^{\prime}\right)\right]_{P Q}^{-1} v_{Q}^{m n^{\prime}}
\end{align}
where $\mathbf{J}_{PQ}$ is the Coulomb interaction projected onto the auxiliary basis, and we have defined
\begin{equation}
    v_{P\mathbf{q}}^{n \mathbf{k}, m \mathbf{k}-\mathbf{q}} = \sum_{pq} C_{pn}(\mathbf{k}) C_{qm}(\mathbf{k}-\mathbf{q}) v_{P\mathbf{q}}^{p\mathbf{k}, q\mathbf{k}-\mathbf{q}}
\label{eq:mochol}
\end{equation}
with
\begin{equation}
    v_{P\mathbf{q}}^{p\mathbf{k}, q\mathbf{k}-\mathbf{q}} = \sum_R (p\mathbf{k}, q\mathbf{k}-\mathbf{q} \mid R\mathbf{q}) \mathbf{J}_{RP}^{-1/2}(\mathbf{q})
\end{equation}
If we first rename $\mathbf{k^\prime} = \mathbf{k}-\mathbf{q} \implies \mathbf{k} = \mathbf{k^\prime} + \mathbf{q}$, and then we are free to redefine $\mathbf{q} \rightarrow -\mathbf{q}$, so that \ref{eq:mochol} becomes
\begin{equation}
    v_{P\mathbf{-q}}^{n \mathbf{k}-\mathbf{q}, m \mathbf{k}} = \sum_{pq} C_{pn}(\mathbf{k}-\mathbf{q}) C_{qm}(\mathbf{k}) v_{P\mathbf{-q}}^{p\mathbf{k}-\mathbf{q}, q\mathbf{k}}
\end{equation}
\section{Integration}
but we know that the bare Coulomb potential projected onto the auxiliary basis is given by
\begin{equation}
    v_{P\mathbf{q}}^{n\mathbf{k}, m\mathbf{k}-\mathbf{q}} = \sum_{pq} C_{pn}(\mathbf{k}) C_{qm}(\mathbf{k}-\mathbf{q}) v_{P\mathbf{q}}^{p\mathbf{k}, q\mathbf{k}-\mathbf{q}}
\end{equation}

To ease notation, some momentum labels are suppressed in the above and following equations (e.g., we will use $b_{P}^{n m}$ to denote $b_{P \mathbf{q}}^{n \mathbf{k}, m \mathbf{k}-\mathbf{q}}$ ). Using Eqs. 19-21, the matrix elements of $W_{0}$ are computed as

\begin{align*}
& \left(n \mathbf{k}, m \mathbf{k}-\mathbf{q}\left|W_{0}\right| m \mathbf{k}-\mathbf{q}, n^{\prime} \mathbf{k}\right) \\
& =\sum_{P Q} b_{P}^{n m}\left[\iint d \mathbf{r} d \mathbf{r}^{\prime} \phi_{P \mathbf{q}}(\mathbf{r}) W_{0}\left(\mathbf{r}, \mathbf{r}^{\prime}, i \omega^{\prime}\right) \phi_{Q(-\mathbf{q})}\left(\mathbf{r}^{\prime}\right)\right] b_{Q}^{m n^{\prime}} \\
& =\sum_{P Q} b_{P}^{n m}\left[\mathbf{J}_{P Q}(\mathbf{q})+\left(\mathbf{J}^{1 / 2} \boldsymbol{\Pi} \mathbf{J}^{1 / 2}\right)_{P Q}(\mathbf{q})+\ldots\right] b_{Q}^{m n^{\prime}}  \tag{22}\\
& =\sum_{P Q} v_{P}^{n m}\left[\mathbf{I}-\mathbf{\Pi}\left(\mathbf{q}, i \omega^{\prime}\right)\right]_{P Q}^{-1} v_{Q}^{m n^{\prime}}
\end{align*}

The 3-center 2-electron integral $v_{P}^{n m}$ between auxiliary basis function $P$ and molecular orbital pairs $n m$ is obtained from an AO to MO transformation of the GDF AO integrals defined in Eq. 15:

\begin{equation*}
v_{P}^{n m}=\sum_{p} \sum_{q} C_{p n}(\mathbf{k}) C_{q m}(\mathbf{k}-\mathbf{q}) v_{P \mathbf{q}}^{p \mathbf{k}, q \mathbf{k}-\mathbf{q}} \tag{23}
\end{equation*}

where $C(\mathbf{k})$ refers to the MO coefficients in the AO basis. $\Pi\left(\mathbf{q}, i \omega^{\prime}\right)$ in Eq. 22 is an auxiliary density response function:

\begin{equation*}
\boldsymbol{\Pi}_{P Q}\left(\mathbf{q}, i \omega^{\prime}\right)=\frac{2}{N_{\mathbf{k}}} \sum_{\mathbf{k}} \sum_{i}^{\mathrm{occ}} \sum_{a}^{\text {vir }} v_{P}^{i a} \frac{\epsilon_{i \mathbf{k}}-\epsilon_{a \mathbf{k}-\mathbf{q}}}{\omega^{\prime 2}+\left(\epsilon_{i \mathbf{k}}-\epsilon_{a \mathbf{k}-\mathbf{q}}\right)^{2}} v_{Q}^{a i} \tag{24}
\end{equation*}


\end{document}